%-------------------------------------------------------------------------------
%	SECTION TITLE
%-------------------------------------------------------------------------------
\cvsection{Work Experience}

%-------------------------------------------------------------------------------
%	CONTENT
%-------------------------------------------------------------------------------
\begin{cventries}

%---------------------------------------------------------
  \cventry
    {SENIOR ARTIFICIAL INTELLIGENCE ENGINEER} % Job title
    {Foghorn Systems} % Organization
    {Pune} % Location
    {Feb 2020 - Present} % Date
    {
      \begin{cvitems} % Description(s) of tasks/responsibilities	
      \item{Decrease Compute time of CNN :  To reduce the compute time of resnet18, designed an architecture that removes less important filters (based on l2 regularization scores) in a convolutional neural network. This architecture is based on soft pruning research paper. Technologies Used: python, pytorch, tensorflowby  }        
	\end{cvitems}
    }
  \cventry
    {SENIOR ARTIFICIAL INTELLIGENCE ENGINEER} % Job title
    {Razorthink} % Organization
    {Bengaluru} % Location
    {Sept 2018 - Dec 2019} % Date
    {
      \begin{cvitems} % Description(s) of tasks/responsibilities	
        \item{Question Answering : This service is used to answer question asked by the end user, using the data provided by the user as a pdf or html document. Did topic modeling using latent dirichlet allocation, this generated topics with word probabilities. Applied a word2vec model on top of it to generate topic vectors for each topic in pdf. Ranking of topics was done by using these topic vectors, these topics were then fed to Bert model for getting answer. Technologies Used: python, nltk, gensim, tqdm}        
        \item{Table Detection : This service is used to detect tables (bordered and borderless) or table like structures in a pdf document. Built and trained a deep learning model for detecting tables. This was built using Faster Rcnn(research paper) and curriculum learning. Pretrained VGG16 net was fine tuned on our documents to achieve a precision of 84 percent.  Technologies Used: python, tensorflow, opencv}        
        \item{Template Detection : This service assigns templates to various documents, two documents belong to the same template if they have similar type of text or structure. Built and trained a siamese network with pretrained VGG16 net to detect whether two images are similar or not. If the first page of two different documents were detected to be similar by the siamese network, they were mapped to the same template in the database. Technologies Used: python, tensorflow, opencv, MongoDb}        
        \item{Carigans : Built a carigan service that converts images of real people to caricatures on the AI Platform of our company. This service was based on a research paper by Microsoft. It makes use of Generative Adversarial networks along with facial recognition to generate cartoon like faces. Gave a presentation on the same (https://www.youtube.com/watch?v=JeWbg83HZes). Technologies Used: python, tensorflow, opencv, scikit-learn}     
	\end{cvitems}
    }

%---------------------------------------------------------
  \cventry
    {Data Scientist} % Job title
    {Nowfloats} % Organization
    {Hyderabad} % Location
    {June 2016 - Sept 2018} % Date
    {
      \begin{cvitems} % Description(s) of tasks/responsibilities
        \item {Developed Wildfire Acquisition Service. It finds potential customers for buying our product who are then targeted to generate revenue. Achieved a high conversion rate of 20 per cent. Used Machine Learning(Logistic Regression) and Pearson Correlation to train the classifier on previous data. Technologies Used: Python, Scikit Learn, Pandas, Matplotlib, MongoDB, MySql  }
        \item {Developed Update Categorization Service. It fetches all the product updates made by the customers on their websites and then categorizes them into offers, discounts, price. Applied a bag of keywords model to classify the updates into categories. Technologies Used: Python, Scikit Learn, Pandas, Matplotlib, MongoDB  }
      \end{cvitems}
    }

\end{cventries}
